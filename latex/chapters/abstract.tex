\setcounter{page}{1}

\chapter*{ABSTRACT}

\vspace{-1cm}
\begin{tabularx}{\linewidth}{@{}p{0.49\linewidth}p{0.51\linewidth}@{}@{}}
\Large {\bf NAME OF INSTITUTION} & \Large Western Institute of Technology \\
\\
\Large {\bf ADDRESS} & \Large Luna Street La Paz, Iloilo \\
\\
\Large{\bf TITLE} & \Large Modular self-navigating unmanned surface drone for marine applications \\
\\
\Large {\bf AUTHOR} & \Large Balasabas, Joseph \\
 & \Large Cordero, John Angelo \\ 
 & \Large Mondragon, John Gee \\
 & \Large Pastrana, Jericho \\
\\
\Large {\bf TYPE OF DOCUMENT} & \Large Research Paper \\
\\
\Large {\bf DATE STARTED} & \Large todo \\
\\
\Large {\bf DATE COMPLETED} & \Large April 2024

\end{tabularx}


\paragraph{} \Large The project is motivated by the desire to explore the potential benefits of modularity coupled with autonomy. Developing a water surface 
            drone leveraging an ESP-32 microcontroller and incorporating autonomous navigation powered by GPS and magnetometer sensors to create an easy
            to use and modular system by minimizing complexities, ensuring smooth operation and simplified troubleshooting. Testing procedures will 
            evaluate the drone's accuracy in reaching designated target positions, providing quantitative insights into its navigational reliability. 
            This project includes information on how to create said drone allowing anyone to reproduce it and use the drone as they see fit as it is 
            modular by nature, some parts can be replaced or swapped out or even add new components to fit users intended use. Given the correct 
            components, mapping lakes and rivers or monitoring ocean conditions for scientific research becomes a much easier task.