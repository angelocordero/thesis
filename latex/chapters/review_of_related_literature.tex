\chapter{Review of Related Literature}

The relevance of unmanned sea drones in the modern era is paramount due to their significant role in enhancing maritime 
operations, surveillance, and data collection while minimizing human risk. These autonomous platforms contribute 
significantly to various domains such as environmental monitoring, security, and underwater exploration. Navigational 
systems play a crucial role in the effectiveness of these drones.

Unmanned sea drones employ diverse navigation systems to ensure precise and effective operation. Inertial Navigation 
Systems (INS) utilize accelerometers and gyroscopes to calculate the drone's position by continuously measuring changes 
in velocity and orientation. Acoustic positioning systems leverage underwater sound waves for localization, particularly 
useful in subsea exploration. Additionally, Doppler Velocity Logs (DVL) provide real-time velocity measurements by 
analyzing the echoes of sound waves off the seafloor. These navigation systems collectively contribute to the autonomy 
and adaptability of sea drones in challenging marine environments. However, the magnetometer and GPS stand out as critical 
components due to their roles in determining heading and ensuring accurate global positioning, respectively, forming a 
foundational framework for the seamless navigation of unmanned sea drones.

The incorporation of Global Positioning System (GPS) and magnetometer technologies significantly contributes to improving
autonomous navigation of water drones in maritime environments. This review explores the current state of research and 
developments in utilizing GPS and magnetometers for precise and reliable open sea navigation. Various studies have explored 
the application of GPS technology in drones.

\section{Prior Arts}
\begin{itemize}
\item {\bf An autonomous boat based synthetic aperture} \\
Earlier innovators such as Silva et al. (2007) introduced the Synthetic Aperture Sonar (SAS) system mounted on an 
autonomous boat, specifically designed for mapping and characterizing shallow water environments. Drawing inspiration 
from Synthetic Aperture Radar (SAR), this SAS system leverages the combined processing of transmitted and received 
signals by a moving probe to generate high-resolution images. Unlike traditional narrow-aperture systems, the proposed 
SAS requires simpler hardware and boasts lower operational costs, making it suitable for various applications like 
bottom tomography, river navigability assessments, and infrastructure inspection. The key innovation lies in utilizing 
a floating platform instead of submersed platforms, enabling accurate position and velocity control through satellite 
navigation systems. This, in turn, enhances the quality of sonar images by providing precise motion compensation for 
the synthetic aperture technique. Silva et al. (2007) further emphasize the tight integration of GPS receivers (with 
carrier phase processing), compass, and inertial navigation system (INS) to achieve positioning and attitude accuracy 
below the acoustic wavelength used. This high-precision data feeds into a back-projection algorithm that generates 
absolute-coordinate images, facilitating seamless integration with other geographic information systems. Overall, the 
presented SAS system offers a promising solution for cost-effective and high-resolution mapping of shallow water 
environments, showcasing the potential of integrating advanced navigation technologies with sonar imaging techniques.

\item {\bf Navigation, guidance, and control of an overactuated marine surface vehicle} \\
Further innovations by Đula Nađ, Nikola Mišković, and Filip Mandić (2017) proposed an optimal control strategy based on a 
PID controller to address course keeping during unmanned surface vessel navigation. The authors employ a combination 
of mathematical modeling, data analysis, simulations, and experiments to validate the effectiveness of their approach. 
Their findings demonstrate that the proposed PID control strategy can ensure course stability for unmanned vessels, 
making it a viable solution for autonomous navigation tasks.

\item {\bf Research on a course control strategy for unmanned surface vessel} \\
Recognizing the limitations of traditional GPS and pre-built maps. Luo et al (2021) address the challenge of precise 
course control for unmanned surface vessels (USVs) in dynamic environments by proposing an optimal control strategy 
based on a PID controller . This approach leverages readily available onboard sensors, including odometry, compass 
readings, and occasional landmark measurements, to maintain course stability under the influence of wind, waves, and 
currents. Through a combination of mathematical modeling, simulations, and experiments, Luo et al. (2021) demonstrate 
effectiveness of their PID control method, showcasing its ability to track desired trajectories even in challenging 
scenarios. Their work offers a promising avenue for robust and adaptable navigation of USVs in real-world conditions, 
laying the groundwork for further exploration of sensor fusion and control algorithms for autonomous maritime operations.

\item {\bf Autonomous urban localization and navigation with limited information} \\
Some studies avoided relying on gps entirely, such as Chipka and Campbell(2018) proposing an algorithm for autonomous 
urban navigation with minimal reliance on traditional GPS or pre-built maps. Chipka and Campbell’s (2018) approach 
utilizes an extended Kalman filter to localize the vehicle based on odometry, compass readings, and occasional landmark 
measurements. Navigation is achieved through a compass-based control law. This method demonstrates promising success 
rates in simulated urban environments under various conditions, suggesting a potential pathway for robust autonomous 
driving even in situations where GPS or detailed maps are unavailable. However, further testing and refinement are 
necessary to assess its real-world feasibility and extend its capabilities to handle more complex scenarios.

\item {\bf Autonomous unmanned merchant vessel and its contribution towards the e-navigation implementation: The MUNIN Perspective} \\
On a broader scale;within the European Union, the MUNIN project develops a concept for an unmanned dry bulk carrier 
during deep-sea voyages..introduced by Burmeister et al. (2014), the MUNIN (Maritime Unmanned Navigation through 
Intelligence in Networks)  project, which investigates the feasibility of autonomous unmanned merchant vessels (UAMVs) 
and their potential alignment with the goals of e-Navigation, focusing on enhancing maritime safety and efficiency. 
While initially appearing to conflict with e-Navigation's emphasis on ship-to-shore cooperation, MUNIN effectively 
bridges this divide through its built-in automation features. Firstly, its advanced sensor module surpasses traditional 
systems in both redundancy and accuracy, potentially mitigating human errors in data interpretation and providing more 
dependable navigation information, a fundamental aspect of e-Navigation. Secondly, MUNIN's centralized architecture 
enables real-time monitoring of the vessel's status and surroundings from shore, enabling proactive intervention and 
improved situational awareness, in line with e-Navigation's objective of enhanced shore-based support. Lastly, MUNIN's 
automation directly addresses e-Navigation's goal of reducing seafarer workload by automating routine tasks, thus 
optimizing operational efficiency and easing crew pressure, potentially fostering a safer maritime environment by 
reducing fatigue. Although acknowledging the ongoing research and development needed for UAMV viability, Burmeister 
et al. (2014) conclude that MUNIN's innovative approach provides valuable insights and advancements that could 
significantly enhance the broader e-Navigation framework.
\end{itemize}

\section{Synthesis}
Unmanned sea drones for modern maritime operations offer enhanced surveillance and data collection while minimizing human 
risk. These platforms use various navigation systems like Inertial Navigation Systems (INS), acoustic positioning, and 
Doppler Velocity Logs (DVL), with GPS and magnetometer technologies playing critical roles in precise positioning and 
heading determination. Recent innovations, like the Synthetic Aperture Sonar (SAS) system on autonomous boats, highlight 
the integration of advanced navigation technologies for high-resolution mapping in shallow water environments. 
Additionally, optimal control strategies based on PID controllers are proving effective in ensuring stability and course 
control for unmanned vessels, showcasing advancements in autonomous maritime navigation.

Moreover, initiatives like the MUNIN project explore the feasibility of an autonomous unmanned merchant vessels (UAMVs) 
aligned with e-Navigation objectives, focusing on safety, efficiency, and reduced seafarer workload. By integrating 
advanced sensor modules and centralized automation, these projects aim to enhance navigation accuracy, shore-based 
monitoring, and operational efficiency. Such endeavors address current navigation challenges and pave the way for safer 
maritime operations by reducing human error and fatigue, offering insights into the future of autonomous maritime 
navigation within the broader framework of e-Navigation.

https://doi.org/10.1016/j.arcontrol.2015.08.005  Navigation, guidance, and control of an overactuated marine surface vehicle. 
            Đula Nađ, Nikola Mišković, Filip Mandić

https://doi.org/10.48550/arXiv.1810.04243 Autonomous Urban Localization and Navigation with Limited Information Jordan Chipka, Mark Campbell

https://doi.org/10.1016/j.enavi.2014.12.002  Autonomous Unmanned\\Merchant Vessel and its Contribution towards the e-Navigation Implementation: The MUNIN Perspective

10.1109/OCEANS.2007.4449358 An Autonomous Boat Based Synthetic Aperture Sonar Sergio Rui Silva; Sergio Cunha; Anibal Matos; Nuno Cruz

10.1088/1742-6596/1948/1/012106 Research on a course control strategy for unmanned surface vessel  Zhigang Luo, Tonghui Qian, Xi Ye, Jiahui Huang and Linwen Yu

\begin{table}[h]
  
\begin{tabularx}{\linewidth}{@{} | m{0.15\linewidth} | p{0.06\linewidth} | p{0.06\linewidth} | p{0.06\linewidth} | p{0.06\linewidth} | p{0.06\linewidth} | p{0.12\linewidth} | m{0.21\linewidth} | }
  \hline
 & Prior art 1 & Prior art 2 & Prior art 3 & Prior art 4 & Prior art 5 & Proponent's study & Remarks \\
 \hline
 GPS / Magnetometer based  & \cmark & \xmark & \cmark & \cmark & \cmark & \cmark & Navigation algorithm is highly reliant on GPS / Magnetometer data \\
 \hline
 3d printed materials & \xmark & \xmark & \xmark & \xmark & \xmark & \cmark & Uses PETG (Polyethylene Terephthalate Glycol-Modified), a material known for its water resistance \\
 \hline
 ESP-32 microcontroller & \xmark & \xmark & \xmark & \xmark & \xmark & \cmark & Utilizes an ESP-32 microcontroller \\
 \hline
 Water surface drone & \cmark & \xmark & \xmark & \cmark & \cmark & \cmark & Built for water surface applications \\
 \hline
 Modular design & \xmark & \xmark & \xmark & \xmark & \xmark & \cmark & Modular components are designed to be easily detached or replaced. This allows for quick upgrades, replacements, or modifications \\
 \hline
\end{tabularx}
\label{table:Synthesis}
\end{table}






